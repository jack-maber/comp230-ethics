% Please do not change the document class
\documentclass{scrartcl}

% Please do not change these packages
\usepackage[hidelinks]{hyperref}
\usepackage[none]{hyphenat}
\usepackage{setspace}
\doublespace

% You may add additional packages here
\usepackage{amsmath}

% Please include a clear, concise, and descriptive title
\title{Should game developers take responsibility for violent actions that occur as a result of their products?}

% Please do not change the subtitle
\subtitle{COMP230 - Ethics Essay}

% Please put your student number in the author field
\author{1606119}

\begin{document}

\maketitle

\abstract{} 

\section{Introduction}
My essay will be on should game developers take responsibility for violent actions that occur as a result of their products? With nearly every violent event that happens around the world, video games are more than likely going to be brought in as one of the main factors for the perpetrators actions, especially after events such as the Columbine Massacre really brought forward the idea that games that contain violence could influence people to commit such horrific acts, but is there a concrete link between the two, and if so, should game developers be put under further scrutiny?

\section{The Causal Link}
There has been countless research papers into whether there is an actual causal link between violence in games and violent actions, with results ranging from ``new video games are ethically affecting people who play them''\cite{lekka2014computer} to ``A causal relationship between violent video games and an individual’s propensity toward violence has not been scientifically determined''\cite{chakraborty2015public}, and thanks to this lack of a definitive answer, video games are still brought up as a factor in many violent acts, especially after the run of shootings such as the Columbine Massacre in the late 90's, after which a groundbreaking lawsuit was brought upon major gaming developers and publishers\cite{familysue}, with a massive emphasis on the fact that ``perpetrators of the Columbine school massacre were gamers''\cite{tavinor2007towards}, bringing both massive attention to violence in games and the developers who make them, even the US government undertook an inquiry into the violent content present in these games, as of a result of which pushed ratings boards such as the ESA to widen ratings\cite{thayer2007analyzing}.\\

 After this point, researchers would bring forward Columbine as  ``hard evidence of digital games’ influence on human behavior''\cite{thayer2007analyzing}, even without that essential causal link needed to back up these claims, normally just citing that they had an interest in games, when it was found at the time that 73 percent of boys aged 9-10 preferred violent games\cite{chakraborty2015public}, but it's immediatly clear that 73 in 100 of this research pool didn't go to commit violent acts, otherwise the world would be in a perpetual state of crime. Then maybe it isn't the developers who should be blamed for these actions, it should be the people commiting the actions themselves, which is what I'm going to cover in my next section. 


\section{Other Factors discussed}
As a large amount of time has passed since the initial media flurry, much more research into the effects of violence in games on players has been carried out, with many of the studies focusing on the ``hyper-realism'' that can be achieved with todays' games, with many theorising that the increased graphical fidelity and use of increasingly complex narritive are a cause to increase access to aggressive/violent concepts\cite{zendle2015higher}, although despite all of this speculation, there is still no evidence to support such a hypothesis, even with all blood and gore removed, it had no effect on the ``aggression-related associations and cognitive processes'' of the player\cite{ashbarry2016blood}. Many of these papers then go on to state that there were many other factors that could contribute to these violent actions, rather than the violent video games, normally citing known mental health or personality issues as the most common cause for the perpertrators actions, the shooting at Virginia Tech in 2007 was undertaken by Seung-Hui Cho, who had previously been diagnosed with major depressive disorder and severe anxiety\cite{seung}, along with writing ``disturbingly violent essays in his English classes'' and ``already made both students and faculty nervous''\cite{massacrelink}, which is much more realistic explanation to why he made these actions, other factors that were also discussed were that of anti-religion and that of their previous education, victims in Columbine where victims were asked "Do you believe in God?", and then shot if they replied positively\cite{gameskill}, along with that in the early 90s, Columbine High School had used ``Death Education'' on students, making them talk about "We talked about what we wanted to look like in our casket.", with one student even admitting that they tried to take their own life after her class\cite{gameskill}. From this, it seems that there were other, much bigger factors at play when these events took place, namely mental health problems and the education system.\\

 With all of these other factors being taken into account really puts into persepective the lack of evidence there is for the link between violent video games and violent events, which is why every case against game developers has been dismissed, as there is no evidence to suggest there is a ``direct correlation between playing “violent” digital games and committing criminal acts such as murder and manslaughter''\cite{thayer2007analyzing}, along with the fact that the media even tried to pin blame onto Marilyn Manson\cite{thayer2007analyzing}, it seems that the people who hold all the power in the media and politics have it out for ``new media'' as it's something they don't understand, and thus perceive it as dangerous, the same happened in the 1950s when comics books came under fire for the same reasons as video games now; it was even found that gamers are more bothered my violence seen in films and TV\cite{bbfc}, showing that the scrutiny that games developers have come under over the year is very much undeserved, it's not the content that causes the problem, it's the people who use it, and a lot of other contributing factors. 




\section{Conclusion}
In conclusion, the developers of violent games do not need to take responsibility for the actions of users of their products as it seems that there were other, much bigger factors at play affecting the perpertrators of these events, from what we've discussed in the last couple of sections, these factors are much more likely the causes of these violent actions as there is much more concrete evidence supporting their contribution over that of video games, along with the lack of actual evidence to support a causal link between violent games and violent actions.  







\bibliographystyle{ieeetran}
\bibliography{references}

\end{document}
