\documentclass{scrartcl}

\usepackage[hidelinks]{hyperref}
\usepackage[none]{hyphenat}

\title{Essay Proposal}
\subtitle{COMP160 - Software Engineering Essay}

\author{Your Name Here}

\begin{document}

\maketitle

\section*{Topic}

My essay will be on: Can playing violent video games affect young peoples' long term mental health?. Thanks to the extremely fast uptake of mobile smart devices, video games have now become easier than ever to gain access to, especially for young people, and with such storefronts as Google Play, the content of said apps normally goes monitored and isn't age restricted by default, so they can gain access to games that may contain violent content. And with this, comes the age old question, of what effect do the games have on young people, many studies have been undertaken on the short term effects, but few have delved into the long term effects it can have, especially whether they can differ between the sexes or if it can take greater affect at a younger age, which is what I want to investigate more in depth in my report.


% Add details as appropriate.

\section*{Paper 1}
% This is an example! Replace the details with a paper relevant to your chosen topic.
\begin{description}
\item[Title:] The Causes and Consequences of Presence: Considering the Influence of Violent Video Games on Presence and Aggression
\item[Citation:] \cite{influence}
\item[Abstract:] The level of presence is likely to influence the effect of media violence. This project examines the causes and consequences of presence in the context of violent video game play. In a between subjects design, 227 participants were randomly assigned to play either a violent or a nonviolent video game. The results are consistent with what would be predicted by social learning theory and are consistent with previous presence research. Causal modeling analyses reveal two separate paths to presence: from individual differences and condition. The first path reveals that individual differences (previous game use and gender) predict presence. Those who frequently play video games reported higher levels of presence than those who play video games less frequently. Males play more games but felt less presence than women. The second path is related to perceived violence: those who perceived the game to be more violent felt more presence than those who perceived less violence in the game. Both of these paths were influenced by frustration with the game, which reduced presence. Those who felt more presence felt more hostility and were more verbally aggressive than those who felt lower levels of presence. Higher levels of presence led to increased physically aggressive intentions. Theoretical and practical implications are discussed.
\item[Web link:] \url{http://ieeexplore.ieee.org.ezproxy.falmouth.ac.uk/document/6797280/}
\item[Full text link:] \url{pdf}
\item[Comments:] Tested 227 People, investigates differences between genders with regards to anger. 
\end{description}

\section*{Paper 2}
\begin{description}
\item[Title:] Brand Memory, Attitude, and State Aggression in Violent Games: Focused on the Roles of Arousal, Negative Affect, and Spatial Presence
\item[Citation:] \cite{brand}
\item[Abstract:] Violent games have increasingly gained their market share in recent video game markets. They have attracted much attention due to their potential effects on users in advertising and aggression. However, little research has investigated such effects considering both user aggression and persuasion mechanism in virtual space. Based on the general aggression model and the presence theory, the current study examines the effects of realistic violence cues (blood and screams) and trait aggression on brand memory, attitude change toward brands in the game, and state aggression through physiological arousal (i.e., Skin conductance level), negative affect and spatial presence. Results show that violence cues affect both arousal and negative affect, and in turn the negative affect changes attitude toward brands negatively and increases the degree of state aggression. Trait aggression enhances presence, and the spatial presence strongly affects brand memory. Results and implications are discussed.
\item[Web link:] http://ieeexplore.ieee.org.ezproxy.falmouth.ac.uk/document/7070241/
\item[Full text link:] http://ieeexplore.ieee.org.ezproxy.falmouth.ac.uk/xpls/icp.jsp?arnumber=7070241
\item[Comments:] Investigates more how the human body reacts to violence. 
\end{description}

\section*{Paper 3}
\begin{description}
\item[Title:] Violent Components and Interactive Mode of Computer Video Game on Player's Negative Social Effect
\item[Citation:] \cite{components}
\item[Abstract:] Abstract:
According to prior studies, aggressive behavior and some other anti-social personality maybe originate from home environment, peer pressure, video game. Violent video game as one part of a violent culture that many children and adults inhabit, it plays more important role in today society. The social effect of violent video game has been discussed by many scientists concerning the juvenile violence. This article briefly reviews the studies of violent video game on players’ negative effect in their personality development, such as aggression, anti-social emotion and action, and so on. Even violent video game is not the only way that contributing to people’s negative behavior. One consistent conclusion can be concluded that violent video game increase players’ aggressive behaviors, aggressive cognitions, psychological arousal, hostility and other negative reactions. However, it's not so much Frustrating. Playing game have positive effects on player, either, and if we use it in a correct way, it may be a suitable way to develop our basic cognitive processing, thinking and problem-solving skills.
\item[Web link:] http://ieeexplore.ieee.org.ezproxy.falmouth.ac.uk/document/5369086/
\item[Full text link:] http://ieeexplore.ieee.org.ezproxy.falmouth.ac.uk/document/5369086/?part=1
\item[Comments:] Investigates the effects on children, which is what my report will focus on. 
\end{description}

\section*{Paper 4}
\begin{description}
\item[Title:] The impact of video games on minors: A review of aggressive and prosocial research
\item[Citation:] \cite{review}
\item[Abstract:] The impact of video games on minors has aroused general concern for decades. The amount of studies addressing violent video games is extensive. Previous research has well documented that playing violent video games is linked to increases in aggressive cognition, aggressive affect, and aggressive behavior, and decreases in prosocial behavior like empathy and helping behavior. In contrast, there has been much less evidence on the effects of prosocial video games. In the last two years, researchers found that exposure to prosocial video games decreased aggressive cognition and increased prosocial behavior. General Learning Model (GLM) can be employed to explain the short-term and long-term effects of video games on the changes of sociality of minors. Further research is needed to uncover the mechanisms.
\item[Web link:] http://ieeexplore.ieee.org.ezproxy.falmouth.ac.uk/document/6056872/
\item[Full text link:] http://ieeexplore.ieee.org.ezproxy.falmouth.ac.uk/stamp/stamp.jsp?tp=&arnumber=6056872
\item[Comments:] Another paper that focuses purely on the ffects on children. 
\end{description}

\section*{Paper 5}
\begin{description}
\item[Title:] The Implicit Aggression in Adolescents: The Priming Effect of Internet Violent Stimulus
\item[Citation:] \cite{bibtex_key}
\item[Abstract:] The goals of the current study used field experiment were to investigate relations between internet violent games and the implicit aggression in adolescents with a sample of 86 middle school students. The results showed that: there is a gender difference when adolescents are affected by implicit aggressiveness that under different priming threshold levels. Compared with girls, supraliminal priming materials are easier to prime boys’ implicit aggressiveness; compared with boys, girls are more likely to be affected by subliminal priming materials. The adolescents who have high level of aggressiveness are easier to be affected by aggressive materials, and significantly increase their implicit aggressiveness in the aggressive priming condition. This result shows that the adolescents who have high aggressive level are more vulnerable to the effects of network violent stimulus and prefer to use violent ways to solve problems.
\item[Web link:] http://ieeexplore.ieee.org.ezproxy.falmouth.ac.uk/document/5474340/
\item[Full text link:] http://ieeexplore.ieee.org.ezproxy.falmouth.ac.uk/document/5474340/?part=1
\item[Comments:] Another paper that focuses purely on the effects on children. 
\end{description}

\section*{Paper 6}
\begin{description}
\item[Title:] The social aspect of human computer activities: An investigation of information technology ethics
\item[Citation:] \cite{ethics}
\item[Abstract:] With high speed broadband connection and new technological breakthrough, there are increasing trend of information technology abuse amongst users. Users seem to neglect laws and guidelines with their intentionally or not committing to the unethical actions. This article presents the understanding of Actions with regards to social and ethical use of Information Technology (IT) within the area of Human Computer Interaction. Initially, 215 literatures were analyzed. From these, fifty literatures that were identified to be highly significant to the social and ethical use of IT were synthesized to develop the conceptual understanding of Actions. From the conceptual understanding, ten Actions that were recommended by the literature to be the most commonly practiced by target user was chosen for the evaluation procedure. In the evaluation procedure, the Actions were organized in a ranking table, and participants were required to rank the Actions according to their common practice. The outcome to this procedure has resulted social networking, piracy and violent computer games being the most popular unethical Actions conducted by the Computer Science students and inappropriate use of camera-phone, digital forgery and on-line gambling being the least conducted. The results provide insights for further research to investigate why the students committed the Actions and how to combat the matter. Other research involving investigation of unethical action in the aspect of social issues in human and computer related activity could also benefit the findings.
\item[Web link:] http://ieeexplore.ieee.org.ezproxy.falmouth.ac.uk/document/6150572/
\item[Full text link:] http://ieeexplore.ieee.org.ezproxy.falmouth.ac.uk/document/6150572/?part=1
\item[Comments:] 
\end{description}

\section*{Paper 7}
\begin{description}
\item[Title:] Computer games and ethical issues
\item[Citation:] \cite{issues}
\item[Abstract:] Summary form only given. Computer games have been exploited in educational procedures, since they help in fostering creativity, in familiarization with technology, and develop problem-solving, logical thinking, communication, and collaborative skills (Sicart 2009). Games is a symptom of our societies and offer learning and strong educational advantages, as they fully motivate and engage students. The term “game” refers to a wide range of activities but, researchers (Juul, 2003, Crawford, 2003) stated that it is difficult to define in terms of necessary and sufficient features. Moreover Grendler (1996, pp. 523) defines games as “consisting of rules that describe allowable player moves, game constraints and privileges (such as ways of earning extra turns) and penalties for illegal (non permissible) actions. Further the rules may be imaginative in that they need not relate to real world-events.” According to Deterding, Sicart, Nacke, O'Hara and Dixon (2011), gamification “is an informal umbrella term for the use of video game elements in non-gaming systems to improve user experience (UX) and user engagement”. Thus, game designers intent to identify and create non-game contexts, products, like points into the game, affecting the players' attitudes and ideas. According to Dang et al. (2007) new video games are ethically affecting people who play them. Additionally, the ethical issues include: violence, rating, education, stereotyping against women, community and addiction. Since computer games usage is increasingly spreading, concern must be placed on the ethical issues that are built in them. There are many games involving violent acts as well as other content related to violence. Thus, many people may believe that playing these types of video games can cause a person to be more violent (Dang et al., 2007). From the education perspective, gaming can be used to teach different things, some positive while others are negative. Educational system must play important role in the ethical improvement of game culture. Responsible game developers need to be informed and take under consideration the research findings concerning the effects of the medium they utilize. Also, game designers and developers need to make knowledgeable decisions for the game content, purpose and goals (Dodig-Crnkovic and Larsson, 2005). Another idea is to design games whose specific goal is to teach ethical principles.
\item[Web link:] http://ieeexplore.ieee.org.ezproxy.falmouth.ac.uk/document/7011160/
\item[Full text link:] http://ieeexplore.ieee.org.ezproxy.falmouth.ac.uk/document/7011160/?part=1
\item[Comments:]  
\end{description}

\section*{Paper 8}
\begin{description}
\item[Title:] Immersive Virtual Video Game Play and Presence: Influences on Aggressive Feelings and Behavior
\item[Citation:] \cite{immersive}
\item[Abstract:] Immersive virtual environment technology (IVET) allows developers to create simulated environments that can engage users in context relevant behaviors and that can produce relatively intense user experiences for purposes such as entertainment (e.g., video games), phobia desensitization, and training. We predicted that playing a violent video game using an IVET platform would lead to increased presence and aggressive feelings and behavior compared to playing on a less immersive desktop platform. The results of two experiments supported this hypothesis. The data suggest that presence mediated the relationship between playing platform and aggressive feelings but not the relationship between playing platform and aggressive behavior. Finally, we explored the utility of using cardiovascular measures within this research paradigm.
\item[Web link:] http://ieeexplore.ieee.org.ezproxy.falmouth.ac.uk/document/6797241/
\item[Full text link:] 
\item[Comments:]  A look at a more immersive experience and how that affects how aggresive the player becomes. 
\end{description}

\section*{Paper 9}
\begin{description}
\item[Title:] On Video Game: Heaven or Hell
\item[Citation:] \cite{heavenhell}
\item[Abstract:] Video games have been the subject of controversy due to the depiction causing addiction and even violent behavior. Although many studies undertaking the examination of video games and the gaming culture deny that games are addictive, a stereotype of the game player as addicted continues to catch on in various domains of academic studies and, with greater effect, in popular culture, news media and governmental rhetoric. The addicted gamers are seen as low-class, proto-violent addicted and dangerous kids, learning to express repressed anger and aggression, sociopathically isolated. In this essay, I am not trying to discuss the details of the arguments, but to discuss the reason why these arguments occurred: Why people denunciate video game addiction? Why addiction is conceived as bad to us?
\item[Web link:] http://ieeexplore.ieee.org.ezproxy.falmouth.ac.uk/document/4407913/
\item[Full text link:] http://ieeexplore.ieee.org.ezproxy.falmouth.ac.uk/document/4407913/?part=1
\item[Comments:]  A look at both video game addiciton and the link between that and violence. 
\end{description}

\section*{Paper 10}
\begin{description}
\item[Title:] On Video Game: Heaven or Hell
\item[Citation:] \cite{heavenhell}
\item[Abstract:] Video games have been the subject of controversy due to the depiction causing addiction and even violent behavior. Although many studies undertaking the examination of video games and the gaming culture deny that games are addictive, a stereotype of the game player as addicted continues to catch on in various domains of academic studies and, with greater effect, in popular culture, news media and governmental rhetoric. The addicted gamers are seen as low-class, proto-violent addicted and dangerous kids, learning to express repressed anger and aggression, sociopathically isolated. In this essay, I am not trying to discuss the details of the arguments, but to discuss the reason why these arguments occurred: Why people denunciate video game addiction? Why addiction is conceived as bad to us?
\item[Web link:] http://ieeexplore.ieee.org.ezproxy.falmouth.ac.uk/document/4407913/
\item[Full text link:] http://ieeexplore.ieee.org.ezproxy.falmouth.ac.uk/document/4407913/?part=1
\item[Comments:]  A look at both video game addiciton and the link between that and violence. 
\end{description}

\section*{Paper 11}
\begin{description}
\item[Title:] 
Violent gaming and player aggression: Exploring the effects of socio-psychological and technology influences
\item[Citation:] \cite{}
\item[Abstract:] Given the high levels of violence in South Africa, there is widespread concern among the public that video game violence may be increasing the risk of aggression in players and consequently leading to violence. Employing the General Aggression Model (GAM), this study set out to examine the influence of socio-psychological factors such as excessive gaming and pathological gaming, and technology factors such as interactive richness on player aggression. The research model was tested using survey data collected from 101 university students. Regression analysis indicates that only pathological gaming explains a small variance in player aggression. Contrary to prior studies that make strong claims that violent video game playing leads to aggressive behaviour, these factors made only a negligible contribution to predicting player aggression in this study. Future research should focus on testing multiple pathways that culminates into player aggression. Violent gaming technologies may not be influencing aggressive behaviour as much as stimulating pre-aggressive individuals already predisposed by other socio-psychological factors. Future research should test the above factors with more contextual based socio-psychological factors, within the individual aggressor's family and close-knit social environment. Intervention and prevention programs may be better served by insights into the aggressor's personal, family, and social context rather than technology related behaviours.
\item[Web link:] https://dl-acm-org.ezproxy.falmouth.ac.uk/citation.cfm?id=2815790&CFID=993586803&CFTOKEN=94178344
\item[Full text link:] 
\item[Comments:]  A look at the impacts on violence and it's connections to real world events
\end{description}


\section*{Paper 12}
\begin{description}
\item[Title:] Analyzing sociocultural perspectives on violence in digital games
\item[Citation:] \cite{}
\item[Abstract:] This article reports the results of a content analysis that tested whether a significant difference in attitude toward violent digital games occurred in the news media as a result of the Columbine school shootings. This article lists attitudinal information about violent game content for more than 30 worldwide news sources, as well as the most frequently mentioned people, institutions, and digital games mentioned by these sources. A one-way ANOVA of authors' attitudes toward violent digital games prior to and after April 20, 1999, as well as ANOVAs testing geographic location, newspaper, and article type, showed no significant attitudinal difference toward violent digital games before and after the Columbine incident. Four cultural themes that relate to the control of violent digital games are also analyzed.
\item[Web link:] https://dl-acm-org.ezproxy.falmouth.ac.uk/citation.cfm?id=1328207&CFID=993586803&CFTOKEN=94178344
\item[Full text link:] 
\item[Comments:]   
\end{description}


\bibliographystyle{ieeetran}
\bibliography{initial_references}

\end{document}
